%%%%%%%%%%%%%%%%%%%%%%%%%%%%%%%%%%%%%%%%%%%%%%%%%%%%%%%%%%%%%%%%%%%%%%%%%%%%%%%%%%%%%%%%%%%%%%%%%%%%%%%
%%													%%
%% 	BAKALÁŘSKÁ PRÁCE -  Zásuvný modul QGIS pro pozemní monitorování radiace			%%
%% 				 Michael Kala							%%
%%													%%
%% pro formátování využita šablona: http://geo3.fsv.cvut.cz/kurzy/mod/resource/view.php?id=775 	%%
%%													%%
%%%%%%%%%%%%%%%%%%%%%%%%%%%%%%%%%%%%%%%%%%%%%%%%%%%%%%%%%%%%%%%%%%%%%%%%%%%%%%%%%%%%%%%%%%%%%%%%%%%%%%% 

\documentclass[%
  12pt,         			% Velikost základního písma je 12 bodů
  a4paper,      			% Formát papíru je A4
  oneside,       			% Oboustranný tisk
  pdftex,				    % překlad bude proveden programem 'pdftex' do PDF
%%%  draft
]{report}       			% Dokument třídy 'zpráva'
%

\newcommand{\Fbox}[1]{\fbox{\strut#1}}

\usepackage[czech, english]{babel}	% použití češtiny, angličtiny
\usepackage[utf8]{inputenc}		% Kódování zdrojových souborů je UTF8

\usepackage[square,sort,comma,numbers]{natbib}

\usepackage{caption}
\usepackage{subcaption}
\captionsetup{font=small}
\usepackage{enumitem} 
\setlist{leftmargin=*} % bez odsazení

\makeatletter
\setlength{\@fptop}{0pt}
\setlength{\@fpbot}{0pt plus 1fil}
\makeatletter

\usepackage[dvips]{graphicx}   
\usepackage{color}
\usepackage{transparent}
\usepackage{wrapfig}
\usepackage{float} 
\usepackage{listings}


\usepackage{cmap}           
\usepackage[T1]{fontenc}    

\usepackage{textcomp}
\usepackage[compact]{titlesec}
\usepackage{amsmath}
\addtolength{\jot}{1em} 

\usepackage{chngcntr}
\counterwithout{footnote}{chapter}

\usepackage{acronym}

\usepackage[
    unicode,                
    breaklinks=true,        
    hypertexnames=false,
    colorlinks=true, % true for print version
    citecolor=black,
    filecolor=black,
    linkcolor=black,
    urlcolor=black
]{hyperref}         

\usepackage{url}
\usepackage{fancyhdr}
%\usepackage{algorithmic}
\usepackage{algorithm}
\usepackage{algcompatible}
\renewcommand{\ALG@name}{Pseudokód}% Update algorithm name
\def\ALG@name{Pseudokód}

\usepackage[
  cvutstyle,          
  bachelor           
]{thesiscvut}


\newif\ifweb
\ifx\ifHtml\undefined % Mimo HTML.
    \webfalse
\else % V HTML.
    \webtrue
\fi 

\renewcommand{\figurename}{Obrázek}
\def\figurename{Obrázek}

%%%%%%%%%%%%%%%%%%%%%%%%%%%%%%%%%%%%%%%%%%%%%%%%%%%%%%%%%%%%%%%%%
%%%%%%%%%%% Definice informací o dokumentu  %%%%%%%%%%%%%%%%%%%%%
%%%%%%%%%%%%%%%%%%%%%%%%%%%%%%%%%%%%%%%%%%%%%%%%%%%%%%%%%%%%%%%%%

%% Název práce
\nazev{ Zásuvný modul QGIS pro pozemní~monitorování~radiace}
{QGIS Plugin for Ground Radiation Monitoring     }

%% Jméno a příjmení autora
\autor{Michael}{Kala}

%% Jméno a příjmení vedoucího práce včetně titulů
\garant{Ing.~Martin~Landa,~Ph.D.}

%% Označení oboru studia
\oborstudia{Geodézie, kartografie a~geoinformatika}{}

%% Označení ústavu
\ustav{Katedra geomatiky}{}

%% Rok obhajoby
\rok{2017}

%Mesic obhajoby
\mesic{červen}

%% Místo obhajoby
\misto{Praha}

%% Abstrakt
\abstrakt 
{Cílem této bakalářské práce je implementace softwarového nástroje umožňujícího plánování optimálních tras pozemního monitorování radiace. Při únicích radioaktivních látek do ovzduší je specializovanými softwary spočtena prognóza šíření radioaktivního mraku. Jedním z produktů této prognózy je také mapa dávkových příkonů záření gama pro zasaženou oblast. Na základě této mapy vytvářený softwarový nástroj určí přibližný odhad dávky záření, kterou obdrží mobilní skupina provádějící měření na dané trase v~postiženém území. V případě překročení hraničních hodnot nástroj pomůže v~přeplánování trasy přes jiné komunikace příp. změny doporučené rychlosti jízdy vozidla tak, aby mobilní skupina nebyla vystavována nebezpečným dávkám.}
{The aim of this bachelor thesis is the implementation of a software tool enabling the management of routes of the ground radiation monitoring. During nuclear disasters, the radioactive substances pollute the environment. Specialized softwares are capable of making a prediction of the spread of the radiation cloud. One of the products of the prediction is also a map of dose rates of the gamma radiation. Based on this map, the created software tool calculates an estimate of the gamma radiation dose which a mobile group doing the field work would obtain on a given route. In case the dose limit value is exceeded, the tool helps to plan changes of the route waypoints using other roads or to modify the recommended speed of the vehicle so the field mobile group would not obtain those values of the dose, that are dangerous or even lethal.}

%% Klíčová slova
\klicovaslova
{GIS, QGIS, zásuvný~modul, python, SÚRO, ionizující záření, radiační ochrana}
{GIS, QGIS, plugin, python, NRPI, ionizing radiation, radiological protection}

%%%%%%%%%%%%%%%%%%%%%%%%%%%%%%%%%%%%%%%%%%%%%%%%%%%%%%%%%%%%%%%%%%%%%%%%

%%%%%%%%%%%%%%%%%%%%%%%%%%%%%%%%%%%%%%%%%%%%%%%%%%%%%%%%%%%%%%%%%%%%%%%%
%% Nastavení polí ve Vlastnostech dokumentu PDF
%%%%%%%%%%%%%%%%%%%%%%%%%%%%%%%%%%%%%%%%%%%%%%%%%%%%%%%%%%%%%%%%%%%%%%%%
\nastavenipdf
%%%%%%%%%%%%%%%%%%%%%%%%%%%%%%%%%%%%%%%%%%%%%%%%%%%%%%%%%%%%%%%%%%%%%%%

%%% Začátek dokumentu
\begin{document}

\catcode`\-=12  % pro vypnuti aktivniho znaku '-' pouzivaneho napr. v \cline 

% aktivace záhlaví
\zahlavi

% předefinování vzhledu záhlaví
\renewcommand{\chaptermark}[1]{%
	\markboth{\MakeUppercase
	{%
	\thechapter.%
	\ #1}}{}}

% Vysázení přebalu práce
%\vytvorobalku

% Vysázení titulní stránky práce
\vytvortitulku

% Vysázení listu zadani
\stranka{}%
	{\includegraphics[scale=0.7]{./pictures/zadani.pdf}}%\sffamily\Huge\centering\ }%ZDE VLOŽIT LIST ZADÁNÍ}%
	%{\sffamily\centering Z~důvodu správného číslování stránek}

% Vysázení stránky s abstraktem
\vytvorabstrakt

% Vysázení prohlaseni o samostatnosti
\vytvorprohlaseni

% Vysázení poděkování
\stranka{%nahore
       }{%uprostred
       }{%dole
       \sffamily
	\begin{flushleft}
		\large
		\MakeUppercase{Poděkování}
	\end{flushleft}
	\vspace{1em}
		%\noindent
	\par\hspace{2ex}
	{Rád bych poděkoval Velkému třesku za vznik vesmíru z nekonečně husté singularity, díky čemuž vznikl i život a já a tato práce. Také bych se nedokázal obejít bez své rodiny a blízkých, díky kterým jsem v posledních měsících nemusel trávit večery sám s myšlenkami o mém počínání. Největší díky bych chtěl věnovat mému vedoucímu práce Ing. Martinu Landovi Ph.D. za to, že mě na rozdíl od přátel k práci velmi motivoval a příkladně jí vedl. Nakonec bych rád poděkoval Mgr. Janu Helebrantovi (SÚRO) za cenné připomínky a vždy bleskové odpovědi na mé dotazy.}
}

% Vysázení obsahu
\obsah

% Vysázení seznamu obrázků
\seznamobrazku

% Vysázení seznamu tabulek
\seznamtabulek

% jednotlivé kapitoly
\chapter{Úvod}
\label{1-uvod}



\chapter{Teoretický základ}
\label{2-teorie}
POPSAT FUKUŠIMU, NĚJAK TAK JAK TO MÁ HELEBRATNT NAPSANÝ U VSTUPNÍCH DAT
POPSAT SAFECAST, DÁT WEBOVKU, CO DĚLAT A JAK TO SOUVISÍ
Tato kapitola se zabývá..
\section{Ionizující záření}
	
Záření (radiace) je proces, při kterém energie prochází prostorem. Typickými příklady záření, se kterými se setkáváme na denní bázi, je sluneční svit, rádiový, televizní nebo i wifi signál. % http://www.world-nuclear.org/information-library/safety-and-security/radiation-and-health/nuclear-radiation-and-health-effects.aspx


Ionizující záření je záření s takovým množstvím energie, že může vyrážet elektrony z atomového obalu a tím látku ionizovat. Tento jev se využívá například v radiologii, lékařském oboru, který ionizující záření používá za účelem diagnóz, terapií a rozvoje vědy. %http://obory.vitalion.cz/radiologie/ %http://www.who.int/ionizing\_radiation/about/what\_is\_ir/en/

\subsection{Druhy ionizujícího záření} %http://astronuklfyzika.cz/JadRadFyzika6.htm
Ionizující záření se dělí na dva druhy:

\begin{itemize}
	\item \textbf{Přímo ionizující}
	
		Kvanta přímo ionizujícího záření nesou elektický náboj a přímo vyrážejí elektrony z atomů. Součástí této kategorie je např. záření $\alpha$ (prudce letící kladná jádra izotopu helia $_{2}He^{4}$ %https://www.cez.cz/edee/content/microsites/nuklearni/k22.htm
		), $\beta^{-}$ (proud elektronů vznikající při přeměně neutronu na proton %https://www.cez.cz/edee/content/microsites/nuklearni/k22.htm)
		), $\beta^{+}$ (proud kladných pozitronů, antičástic k elektronům) %https://www.cez.cz/edee/content/microsites/nuklearni/k22.htm)
		 atd.
		 
	\item \textbf{Nepřímo ionizující}
	
		Nepřímo ionizující záření předává svou kinetickou energii nabitým látkovým částicím, které pak látku ionizují. Kvanta tohoto záření tedy nejsou elektricky nabita. Do této kategorie se řadí především rentgenové a $\gamma$ záření (elektromagnetická záření s velmi nízkou vlnovou délkou). %http://astronuklfyzika.cz/JadRadFyzika6.htm	
\end{itemize}

\subsection{Fyzikální jednotky} %http://atominfo.cz/2012/05/sievert-becquerel-rentgen-jak-merime-radioaktivitu/

\begin{itemize}
	\item \textbf{Dávka, dávkový příkon}
	
		V jaderné fyzice se jako základní jednotka používá becquerel [Bq] patřící mezi odvozené jednotky soustavy SI. Vyjadřuje střední počet radiokativních přeměn za sekundu. Tato jednotka ovšem neříká nic o druhu záření, jeho biologickému účinku, energii atd. Pro popis ionizujícího záření jsou zavedeny veličiny, které slouží jako charakteristiky jeho účinků na různé látky. Nejdůležitější z těchto veličin je tzv. dávka, která má za jednotku gray [Gy] a patří mezi základní jednotky SI. Fyzikální rozměr jednotky gray je joule na kilogram [J/kg]. Častá veličina je dále tzv. dávkový příkon, tedy přírust dávky v čase [Gy/s]. 
		
	\item \textbf{Ekvivalentní dávka, ekvivalentní dávkový příkon}

	 	V předchozím bodě zmíněné veličiny nezohledňují všechny účinky působení záření na živou hmotu. Proto byly zavedeny radiobiologické veličiny, které toto zohledňují. Je to ekvivalentní dávka, která je vypočtena z dávky přenásobené tzv. jakostním činitelem Q závisejícím na typu a energii záření. Jeho hodnota je doporučovaná mezinárodní komisí radiologické ochrany\footnote{V dokumentu ICRP Publication 103 dostupném na http://www.sujb.cz/fileadmin/sujb/docs/radiacni-ochrana/ICRP103\_dokument.pdf}. Například u gama záření je Q = 1. Ekvivalentní dávka má za jednotku sievert [Sv], ekvivalentní dávkový příkon pak sievert za jednotku času [Sv/s].
	 	
	\item \textbf{Shrnutí jednotek}
	
		\begin{table}[h!]
			\centering
			\caption{Fyzikální jednotky ionizujícího záření}
			\label{tab:tabulkaJednotek}
			\begin{tabular}{|c|c|c|}
				\hline
				\textbf{Název}              & \textbf{Jednotka}  & \textbf{Značení} \\ \hline
				Dávka                       & gray               & {[}Gy{]}         \\ \hline
				Dávkový příkon              & gray za sekundu    & {[}Gy/s{]}       \\ \hline
				Ekvivalentní dávka          & sievert            & {[}Sv{]}         \\ \hline
				Ekvivalentní dávkový příkon & sievert za sekundu & {[}Sv/s{]}       \\ \hline
			\end{tabular}
		\end{table}
\end{itemize}

\subsection{Zdroje ionizujícího záření} 
Zdroje ionizujícího záření mohou být přírodní a umělé. Největší ozáření obyvatelstva způsobují zdroje přírodní, přestože pozornost je věnována především zdrojům umělým. %http://fbmi.sirdik.org/4-kapitola/41.html

\begin{itemize}
	\item \textbf{Přírodní zdroje záření}
	
			Tyto zdroje tvoří tzv. přírodní pozadí. Přírodní zdroje jsou dále rozděleny na dvě kategorie, kosmické záření a přírodní radionuklidy. %http://fbmi.sirdik.org/4-kapitola/42
			Množství kosmického záření se odvíjí od nadmořské výšky a zeměpisné šířky kvůli působení zemského magnetického pole na dráhu nabitých částic. Například mezi 30$^{o}$ a 60$^{o}$ jižní resp. severní šířky je intenzita záření příbližně o 10\% vyšší než na rovníku a magnetických pólech. %http://fbmi.sirdik.org/4-kapitola/42/421.html
			Zdroje přírodních radionuklidů jsou především horniny. Intenzita záření  se odvíjí od původů jednotlivých hornin. Pro ilustraci, vyvřelé horniny vykazují vyšší aktivitu než horniny metamorfované. %http://fbmi.sirdik.org/4-kapitola/42/422.html
	
	\item \textbf{Umělé zdroje záření} %http://fbmi.sirdik.org/4-kapitola/43.html
	
			Za umělé zdroje záření jsou považovány takové zdroje, které způsobují ozáření při činnostech s nimi, dále takové zdroje, které souvisí s lékařskými zákroky. Běžně se vedle lékařského ozáření další zdroje podílí na ozáření člověka pouze minimálně. Dalšími zdroji jsou radionuklidy nacházející se v životním prostředí pocházející ze spadu po mimořádných jaderných haváriích (poškození jaderného zařízení) nebo po zkouškách jaderných zbraní. Radionuklidy, které se dostaly do ovzduší, se dostávají na povrch ve formě suchého nebo mokrého spadu s deštěm, kde kontaminují vodu a potravu. %http://fbmi.sirdik.org/4-kapitola/43/432.html		
			
\end{itemize}

Podrobnější popis zdrojů ionizujícího záření by byl nad rámec této bakalářské práce, proto nebude dále rozebírán. 

\subsection{Biologické účinky ionizujícího záření}
Pro stanovení kritérií a principů radiační ochrany obyvatelstva a pracujících, kteří přicházejí se zdroji ionizujícího záření více do kontaktu, je potřeba vědět, jak ionizující záření působí na lidský organismus. Z těchto kritérií je dále je odvozen systém limitování dávek (viz. podkapitola ???). 
%https://www.sujb.cz/radiacni-ochrana/oznameni-a-informace/strucny-prehled-biologickych-ucinku-zareni/

Jak již bylo stručně popsáno, ionizující záření (radiace) způsobuje ionizaci atomů. Ta může dále vést k chemickým reakcím, fyzikálním změnám a v případě živých tkání k biochemickým změnám. Tyto změny mohou vést k poškození organismu nebo i k jeho úmrtí. Účinek radiace na organismus je rozdělen na 4 následující etapy: %http://astronuklfyzika.cz/RadiacniOchrana.htm#2

\begin{enumerate}
	\item \textbf{Fyzikální stádium}
	
		Fyzikální stádium je primární proces, při kterém dochází k ionizaci atomů (toto vede k narušení chemických vazeb mezi atomy a molekulami). Při dávce 1Gy (jednotky dávky záření budou rozebrány v kapitole ??) se v objemu každé ozářené buňky o typické velikosti 10$\mu$m vytváří 10$^5$ ionizací. Tento proces trvá jen cca 10$^{-16}$ - 10$^{-14}$s.
		
	\item \textbf{Fyzikálně-chemické stádium}
	
	Sekundárním procesem je fyzikálně-chemické stádium, při kterém dochází k disociaci molekul (rozklad na kladně a záporně nabité částice) a vzniku volných radikálů (vysoce reaktivních částic). Tento proces je podobně jako proces předchozí velmi rychlý. Trvá přibližně 10$^{-14}$ - 10$^{-10}$s.	 
	
	\item \textbf{Chemické stádium}
	
	Produkty předchozích stádií reagují s důležitými organickými molekulami a mění jejich složení a funkci. Například zlomy řetězců v molekule DNA jsou řazeny mezi typické poruchy. Trvání tohoto stádia ovlivňuje transportní doba reaktivních složek z místa svého vzniku do místa napadené biomolekuly v rozmezí od tisícin sekundy do řádově jednotek sekundy.
	
	\item \textbf{Biologické stádium}
	
	Popsané molekulární změny mohou vyústit ve funkční a morfologické změny v buňkách, orgánech a poté i celkově v organismu. Trvání této fáze se pohybuje od jednotek sekund (buňky) až po několik let (organismus). Kdy se biologické stádium projeví záleží na množství dávky záření. Při nízkých dávkách se může projevit až za několik desítek let, kdežto naopak při vysokých dávkách již během desítek minut. 
\end{enumerate}

Lidský organismus má omezenou schopnost opravy poškozených molekul buněk. Pokud však dávka překročí určitou mez, buňky uhynou a vzniká tzv. nemoc z ozáření. 	% všechno až sem: http://astronuklfyzika.cz/RadiacniOchrana.htm#2
Nemoc z ozáření může být rozdělena na 2 kategorie: %http://www.priznaky-projevy.cz/traumatologie/nemoc-z-ozareni-radiacni-syndrom-priznaky-projevy-symptomy

\begin{itemize}
	\item \textbf{Akutní nemoc z ozáření}
	
		Akutní nemoc z ozáření je způsobena jednorázovým ozářením. Prvotními příznaky je nevolnost, zvracení a průjmy. Pokud dávka ozáření překročí hodnotu přibližně 4 Sv (viz. kapitola ????), nastupuje tzv. střevní forma, kterou doprovází krvavé průjmy a minerální rozvrat. Poté přichází období latence (prodlevy), jehož délka trvání závisí na množství absorbované dávky. Po uplynutí latentní fáze nastupuje tzv. dřeňová forma, kdy dojde k zhroucení krvetvorby a imunitních mechanismů. Nemoc v této fázi dále způsobuje sepsi, sterilitu, u těhotných žen potrat atp. Pokud dojde k absorbaci dávek záření vyšších než 10 Sv, dochází k nevratnému poškození buněk centrálního nervového systému a později nastává smrt.
		
	\item \textbf{Chronická nemoc z ozáření}
	
		Tato forma nemoci z ozáření se rozvíjí při dlouhodobém působení malých dávek ionizujícího záření. Dále se dělí na 3 fáze. První z nich je fáze nespecifických obtíží způsobující nespavost, bolesti hlavy, pokles bílých krvinek, zažívací obtíže atd. Další z fází je fáze výrazné symptomatologie, kde se stupňují bolesti hlavy, dochází k poruchám motoriky, k chronickým průjmům, váhovým úbytkům atd. Dochází k poškození centrálního nervového systému, což doprovází zhoršený sluch a zrak. Následuje poslední fáze nezvratného poškození. Přestávají fungovat rozmnožovací orgány, dochází k poškození srdce, ledvin, jater, dále se na kůži a sliznici tvoří vředy atd. 
		%http://www.priznaky-projevy.cz/traumatologie/nemoc-z-ozareni-radiacni-syndrom-priznaky-projevy-symptomy
\end{itemize}

\section{Radiační ochrana} %https://www.suro.cz/cz/radiacni-ochrana
Cílem radiační ochrany je zajistit ochranu obyvatelstva před účinky ionizujícího záření a zároveň umožnit z těchto účinků co vytěžit co největší přínos (v radiologii, v jaderné energetice atp.) 
		







%%asim.utia.cas.cz/reporty/2011/HZS_FINAL_po%20zkraceni.pdf

%%sujb.cz/fileadmin/sujb/docs/zpravy/vyrocni_zpravy/ceske/VZ_SUJB_2015_FIN_cast_II.pdf

%%sujb.cz/aktualne/detail/clanek/zona-2015-cviceni-simulovane-havarie-v-temeline-1

Zeptat se, čím SÚRO monitoruje pozemní radiaci (jakým vehiklem, nějak ho popsat)

https://www.suro.cz/cz/vyzkum/vysledky/metodiky/Metodika%20detekce%20radioaktivnich%20latek%20na%20zasazenem%20uzemi.pdf/view

https://www.suro.cz/cz/vyzkum/vysledky/mobilni-a-stacionarni-radiacni-monitorovaci-systemy-nove-generace-pro-radiacni-monitorovaci-site-mostar

Mobilní a stacionární radiační monitorovací systémy nové generace pro radiační monitorovací sítě (MOSTAR)

Metody radiační ochrany, upřesnit k čemu se to bude používat, nerad bych tam psal nesmysly.
\chapter{Použité technologie}
\label{3-technologie}

V~této kapitole budou popsány technologie použité pro tvorbu
softwarového nástroje, jež je předmětem této bakalářské
práce. Implementace byla provedena v~programo\-vacím jazyku Python
s~využitím grafického frameworku PyQt a QGIS API (rozhraní pro
programování aplikací).

\section{Python}
\begin{figure}[H] \centering
      \includegraphics[width=100pt]{./pictures/python.png}
      \caption[Python logo]{Python logo (zdroj: Wikimedia Commons)}
      \label{fig:python}
\end{figure}
  
Python je objektově orientovaným skriptovacím jazykem provozovatelným
téměr na každé platformě. Oceňován programátory (především
začátečníky) je pro svou jednoduchou a velice efektivní syntax,
například také díky dynamickému typování (nepožaduje specifikaci
datového typu u~proměnných). Dále jeho objektový model podporuje
polymorfismus, přetěžování operátorů a vícenásobnou dědičnost. Hlavním
%%% ML: tomu nerozumim, jak souvisi "skriptovaci" jazyk s integraci C?
důvodem, proč je zařazován mezi skriptovací jazyky je jeho integrace
s~jazykem C. Díky knihovnám Python/C API lze z~programů v~Pythonu
volat kód psaný v~C nebo naopak, pro aplikace psané v~C je možné
integrovat interpret Pythonu.  Python bývá označován jako "spustitelný
pseudokód" - syntaxe se snaží vyhnout složitým zápisům a znakům ($\$,
<<, \&\&, ?$), inspiruje se v~matematice zápisem abstraktních
algoritmů.\cite{learningPython}

Jazyk Python byl navržen koncem 80. let nizozemským počítačovým
programátorem Guido van Rossumem. První verzi (0.9.0) uveřejnil
začátkem roku 1991. Hlavní principy Pythonu, které zakladatel jazyka
prosazuje, byly shrnuty do podoby 20 aforismů (\textit{"Zen of Python,
by Tim Peters"}), např.:

\begin{itemize}

	\item \textit{Na čitelnosti záleží. (Readability counts.)}
			
	\item \textit{Chyby by nikdy neměly projít bez
povšimnutí. Jedině pokud nejsou záměrně zamlčeny. (Errors should never
pass silently. Unless explicitly silenced.)}
		
	\item \textit{Měl by existovat jeden - a pokud možno pouze
jeden - zřejmý způsob jak to udělat. (There should be one - and
preferably only one - obvious way to do it.)}
\end{itemize}

Pro výpis anglického originálu Zenu stačí do konzole Pythonu napsat \\
\texttt{$>>>$~\textbf{import}~this}. Od roku 2008 je vyvíjena řada
Pythonu 3, která se snaží naplnit právě poslední ze zmiňovaných
aforismů, kdy důraz je kladen na odstranění duplicitních programových
konstrukcí a modulů. Python 3 není plně zpětně kompatibilní s~řadou 2,
softwary napsané a využívající Python 2 postupně přecházejí na jeho
novou verzi (i QGIS s~připravovanou verzí QGIS 3.0). Pro zajímavost,
jazyk byl pojmenován podle britské komediální skupiny Monty
Python. \cite{pythonHistory}

\section{Qt}

\begin{figure}[H] \centering
      \includegraphics[width=40pt]{./pictures/qt.png}
      \caption[Qt logo]{Qt logo (zdroj: Wikimedia Commons)}
      \label{fig:python}
\end{figure}

Qt je aplikační vývojový framework umožňující multiplatformní produkci
aplikací s~grafickým uživatelským rozhraním (\zk{GUI}). Qt dále nabízí
mnoho multiplatformních vývojových nástrojů pro usnadnění vývoje
aplikací pomocí této technologie. Vedle vývojového prostředí Qt
Creator (umožňuje kompletní vývoj aplikací) a dalších, je to
samostatný nástroj pro tvorbu \zk{GUI} Qt Designer, který byl využíván
při tvorbě vytvářeného softwarového nástroje. \cite{qt}

\section{PyQt} PyQt je propojení Pythonu jako programovacího jazyka a
Qt jako frameworku poskytujícího nástroj pro tvorbu
\zk{GUI}. Kombinace Pythonu a Qt umožňuje vyvíjet aplikace, které jsou
kompatibilní s~většinou používaných platforem a otevírá dveře
začátečníkům díky jednoduchosti, přehlednosti a síle Pythonu. PyQt je
používán pro vývoj všech druhů grafických aplikací, od programů pro
vedení účetnictví přes vědci užívané softwarové nástroje pro tvorbu
vizualizací až po počítačové hry. \cite{rapidPyQt}

\section{QGIS}

\begin{figure}[H] \centering
      \includegraphics[width=100pt]{./pictures/qgis.png}
      \caption[QGIS logo]{QGIS logo (zdroj: QGIS.org blog)}
      \label{fig:qgis}
\end{figure}
 
QGIS je multiplatformní geografický informační systém (\zk{GIS})
vyvíjený jako Open Source šířený pod Obecnou veřejnou licencí GNU
(\zk{GNU GPL} - GNU General Public License). \zk{GNU GPL} zaručuje
svobodu jeho sdílení a úprav, které vedou k~implementaci nových
funkcionalit a k~jeho zdokonalení. To z~něj činí mocný nástroj
používaný ve veřejném i soukromém sektoru. QGIS je psán
v~programovacím jazyku C++, jeho grafické uživatelské rozhraní je
postaveno na knihovně Qt. Projekt QGIS započal v~roce 2002, verze 1.0
byla uveřejněna roku 2009. QGIS disponuje nepřeberným množstvím
zásuvných modulů (pluginů) rozšiřujících funkčnost softwaru. Pluginy
jsou programovány v~jazyku C++ nebo Python. \cite{masteringQgis}

\chapter[Zásuvný modul]{Zásuvný modul \includegraphics[scale=0.65]{./pictures/ikonka.png}\footnote{Tady ocitovat ikonku, že je od kolegy ze SÚRA}}
\label{4-plugin}

V následujícím textu bude popsán postup tvorby nového softwarového nástroje \textit{Ground radiation monitoring} a jeho funkcionalita. Při vývoji nástroje bylo čerpáno z doporučené literatury SEM DÁT TY KNÍŽKY ZE ZADÁNÍ. 

\section{Zadání}
Zadáním bakalářské práce bylo vytvoření softwarového nástroje, který ze vstupní interpolované mapy dávkových příkonů extrahuje data do naplánovaných tras monitorování a vypočítá obdrženou dávku záření gama při zadané rychlosti. Nástroj dále vypočte jednoduché statistiky, maximální a průměrný dávkový příkon, délku trasy, čas a~kumulativní dávku v~určitých zadaných intervalech.

\subsection{Vstupní data}
\begin{enumerate}
	\item \textbf{Interpolovaná mapa dávkového příkonu} \\
	Je vytvořena v rastrovém formátu, který je podporován knihovnou GDAL. Obsahuje hodnoty dávkového příkonu v daných jednotkách. (Plugin umožňuje volit typ jednotek). Mapa je v souřadnicovém systému WGS84 EPSG:4326.
	\item \textbf{Trasa monitorování} \\
	Je vytvořena ve vektorovém formátu, který je podporován knihovnou OGR. Trasy mohou být generované pomocí plánovačů tras, např. společnosti Google, Inc. Trasa monitorování je taktéž v souřadnicovém systému WGS84 EPSG:4326. 
\end{enumerate}

			\begin{figure}[H]
    			\centering
      			\includegraphics[scale=0.7]{./pictures/ukazka_vstupnich_dat.png}
      				\caption[Ukázka vstupních dat]{Ukázka vstupních dat}(zdroj: co sem napsat?)
     				\label{fig:vstup}
  			\end{figure}
  			
\subsection{Výstupní data}
\begin{enumerate}
	\item \textbf{Soubor se zprávou o výpočtu} \\
	Soubor se zprávou o výpočtu v textovém formátu (\textit{.txt}) obsahuje následující informace o výpočtu (v anglickém jazyce):
		\begin{itemize}
			\item čas vytvoření zprávy (\textit{report created})
			
			\item informace o trase (\textit{route information})
			\begin{itemize}
				\item název trasy (\textit{route})
				\item zadaná rychlost v km/h (\textit{monitoring speed (km/h)})
				\item celkový čas monitorování v h:mm:ss (\textit{total monitoring time (h:mm:ss)})
				\item celková vzdálenost v km (\textit{total distance (km)})
			\end{itemize}
			
			\item informace o části trasy bez dostupných dat (v místech, kde trasa přesahuje mapu dávkového příkonu) (\textit{no data})
			\begin{itemize}
				\item čas (\textit{time})
				\item vzdálenost v km (\textit{distance (km)})
			\end{itemize}
			
			\item statistické hodnoty (\textit{radiation values (estimated)})
			\begin{itemize}
				\item maximální dávkový příkon v $\mu$Sv/h (\textit{maximum dose rate ($\mu$Sv/h)})
				\item průměrný dávkový příkon v $\mu$Sv/h (\textit{average dose rate ($\mu$Sv/h)})
				\item celková dávka v $\mu$Sv (\textit{total dose ($\mu$Sv)})
			\end{itemize}
			
			\item nastavení (\textit{plugin settings})
			\begin{itemize}
				\item jednotky dávkového příkonu vstupní mapy (\textit{input raster units})
				\item vzdálenost mezi body navzorkované trasy v m (vysvětleno v kapitole NAPSAT KDE JE VYSVĚTLENÝ VZORKOVÁNÍ (\textit{distance between track vertices (m)})
			\end{itemize}
			
		\end{itemize}
	
			\begin{figure}[H]
    			\centering
      			\includegraphics[scale=0.8]{./pictures/report.png}
      				\caption[Ukázka zprávy o výpočtu]{Ukázka zprávy o výpočtu}(zdroj: co sem napsat?)
     				\label{fig:report}
  			\end{figure}
  			
	\item \textbf{Soubor trasy} \\
	Soubor trasy obsahuje bodovou vrstvu ve formátu Esri Shapefile (\textit{.shp}) s body trasy navzorkované dle zadání uživatele (bude vysvětleno v kapitole NAPSAT KDE JE VYSVĚTLENÝ VZORKOVÁNÍ) s následujícími atributy:
		\begin{itemize}
			\item dávkový příkon
			\item kumulativní čas
			\item časový interval mezi body
			\item kumulativní dávka
		\end{itemize}
			\begin{figure}[H]
    			\centering
      			\includegraphics[scale=0.8]{./pictures/atributova_tabulka.png}
      				\caption[Výřez atributové tabulky]{Výřez atributové tabulky}(zdroj: co sem napsat?)
     				\label{fig:atributova_tabulka}
  			\end{figure}
  	
  	\item \textbf{Soubor s údaji o trase (volitelné)} \\
  	V souboru s údaji o trase ve formátu CSV (\textit{.csv}, hodnoty oddělené čárkou) jsou obsaženy stejné hodnoty jako v atributové tabulce navzorkované trasy. Navíc soubor obsahuje souřadnice bodů. Vytvoření souboru je volitelné. 
  			\begin{figure}[H]
    			\centering
      			\includegraphics[scale=0.8]{./pictures/csv.png}
      				\caption[Výřez ze souboru s hodnotami oddělenými čárkou]{Výřez ze souboru s hodnotami oddělenými čárkou}(zdroj: co sem napsat?)
     				\label{fig:csv}
  			\end{figure}	
\end{enumerate}

\section{Popis kódu(pracovní název)}
\subsection[Plugin Builder]{Plugin Builder \includegraphics[scale=0.1]{./pictures/plugin_builder.png}}
K vytvoření základu softwarového nástroje byl použit zásuvný modul Plugin Builder dostupný z oficiálního QGIS repozitáře.\footnote{Dostupné z \url{https://plugins.qgis.org/plugins/pluginbuilder/}} Tento modul pochází z dílny organizace GeoApt LLC, jež se zabývá volně šiřitelným GIS. Po zadání základních informací (název modulu, základní popis, autor, požadovaná verze QGIS, odkazy a další údaje o repozitáři apod.) Plugin Builder vytvoří kostru nového zásuvného modulu. Tato kostra zajišťuje jeho základní funkcionalitu, tedy zobrazení okna a jeho vypnutí, potažmo tlačítka \texttt{OK | Cancel}, pokud není nastaveno, že okno zásuvného modulu bude "přichycovací". 

\subsection{Popis souborů}
Celý zásuvný modul \textit{Ground radiation monitoring} tvoří několik souborů dohromady tvořících balíček, který zajišťuje spustitelnost a funkcionalitu modulu. Některé soubory zde budou prezentovány. Funkcionalita zásuvného modulu zajišťující řešení zadání bakalářské práce je uložena v posledních dvou jmenovaných souborech, které obsahují modifikace a bloky kódu zaručující výsledky práce. %tohle napsat nějak lépe kámo

\begin{itemize} %to mam z Mastering QGIS, ještě že mastruju
	\item \textbf{\_\_init\_\_.py} \\ 
		Tento soubor slouží pro základní inicializaci modulu.
		 
	\item \textbf{metadata.txt} \\
		Tento textový soubor obsahuje informace o zásuvném modulu čtené Správcem zásuvných modulů. Vedle údajů jako je jméno autora a název modulu je zde například také údaj o požadované verzi QGIS, pro kterou je modul naprogramován. Správce pak tento údaj porovná s verzí QGIS a pokud dojde ke konffliktu, tak vypíše chybovou hlášku a modul nenaimportuje.
	
	\item \textbf{Makefile} \\
		V tomto souboru se nachází set instrukcí např. pro zkompilování dokumentace nebo souboru \textbf{resources.qrc} (zkompilovaná verze je \textbf{resources.py}), který informuje Qt jak naložit s ikonou modulu.
		
	\item \textbf{plugin\_upload.py} \\
		Tento soubor pro nahrání modulu do QGIS repozitáe zásuvných modulů.

	\item \textbf{ground\_radiation\_monitoring.py} \\
		Tento soubor slouží pro implementaci zásuvného modulu do prostředí QGIS. Obsahuje třídu \texttt{GroundRadiationMonitoring}, jejíž zásadními metodami jsou \texttt{add\_action} - metoda načítající ikonu modulu (včetně názvu) do nástrojové lišty QGIS a do menu, tedy přidává tlačítko na spuštění. Dále jsou to metody \texttt{onClosePlugin} a \texttt{unload}, které se starají o destrukci modulu.

	\item \textbf{ground\_radiation\_monitoring\_dockwidget.py} \\
		Tento soubor zajišťuje propojení s grafickým rozhraním, které je vytvořené v souboru \textbf{ground\_radiation\_monitoring\_base.ui} pomocí prostředí QT Designer. Obsahuje třídu \texttt{GroundRadiationDockWidget}, ve které jsou implementovány metody pro načítání vstupních dat, čtení údajů zadaných uživatelem, především také pro spuštění (a případné přerušení) procesu výpočtu a práci s výstupním souborem trasy (pokud si to uživatel přeje, vrstva s trasou může být načtena do QGIS). V případě chyby při zadání vstupních parametrů (např. zadání textu do pole, do kterého má být zadané číslo nebo výběr výstupního souboru, do kterého je zákaz zápisu) je uživatel upozorněn chybovým hlášením.     
	
	\item \textbf{ground\_radiation\_monitoring\_computation.py} \\
		V tomto souboru probíhá samotný výpočet dle uživatelsky zadaných dat a vstupních parametrů. Obsahuje třídu \texttt{GroundRadiationMonitoringComputation}, která je implementována jako samostatné výpočetní vlákno. Toto má za výhodu, že výpočet probíhá na pozadí, tedy že s QGIS se dá pracovat dále nezávisle na probíhajícím procesu, což je nezbytné vzhledem k jeho někdy dlouhému trvání (v závislosti na vstupních proměnných). V této třídě jsou vedle výpočetních metod také metody pro vytváření výstupních souborů. Třída během výpočtu komunikuje s hlavním vláknem (s třídou \texttt{GroundRadiationMonitoringDockWidget}) přes signály, pomocí kterých informuje o postupu výpočtu, který je zobrazován v ukazateli průběhu výpočtu.
	
\end{itemize}

\subsection{Vzorkování linie}


\section{Testování}
Popsat testovací data (použily se tyhle, k tomu obrázky).


\chapter{Závěr}
\label{5-zaver}

Hlavním cílem této bakalářské práce byla implementace zásuvného modulu do QGIS rozšiřujícího softwarové vybavení Státního ústavu pro radiační ochranu v.v.i. (i jiných institucí). Nástroj poslouží jako další prostředek k~zrychlení reakční doby při cvičeních radiačních havárií příp. při samotných radiačních haváriích a především k~zvýšení bezpečnosti mobilních skupin pracujících přímo v~terénu. Dále si práce kladla za cíl seznámit čtenáře s~dopady ionizujícího záření na lidský organismus a s~metodou pozemního snímání radiace. 

Vytvářený nástroj ze vstupní interpolované mapy dávkových příkonů gama záření vyextrahuje data na dané trase a spočte několik základních statistik. Zejména se jedná o~celkovou dávku gama záření, kterou mobilní skupina na trase obdrží. V~případě překročení mezních hodnot obdržené dávky, které již ohrožují na zdraví, je možné pomocí dalších výsledků výpočtu vytipovat místa na trase, kterým se vyhnout a tedy trasu monitorování modifikovat. 

Nástroj byl s~pomocí návrhů a požadavků na úpravy od pracovníků \zk{SÚRO} vytvořen na míru tak, aby splnil očekávání a byl zároveň co nejvíce intuitivní. Nástroj byl vytvořen v~anglickém jazyce s~vidinou jeho využití i v~jiných než českých institucích. Zároveň byl v~angličtině vytvořen návod na jeho používání. 


\section{Licence a dostupnost zásuvného modulu}
Zásuvný modul byl vytvořen pod licencí \zk{GNU GPL}, kterou podědil po QGIS knihovnách, jež byly k~implementaci využity. Zásuvný modul je volně dostupný z~CTU GeoForAll Lab QGIS repozitáře\footnote{\texttt{https://ctu-geoforall-lab-projects.github.io/bp-kala-2017/}}. V~budoucnu je dále v~plánu začlenění zásuvného modulu do oficiálního QGIS repozitáře.


 





% Vysázení seznamu zkratek

\begin{seznamzkratek}{ABCDE}

	\novazkratka{SÚRO}
	      {SÚRO}
	      {Státní ústav radiační ochrany}
	      
	\novazkratka{PSF}
		  {PSF}
	      {Python Software Foundation}

	\novazkratka{GIS}
	      {GIS}
	      {Geografický informační systém}

      \novazkratka{OSGeo}
	      {OSGeo}
	      {Open Source Geospatial Foundation}
	         
	  \novazkratka{GUI}	
	      {GUI}
	      {Grafické uživatelské rozhraní (Graphical user interface)}
	      
	  \novazkratka{CSV}	
	      {CSV}
	      {Hodnoty oddělené čárkami (Comma-separated values)}
	      
	  \novazkratka{WGS84}	
	      {WGS84}
	      {Světový geodetický systém 1984 (World Geodetic System 1984)}

	  \novazkratka{GPL}	
	      {GPL}
	      {Všeobecná veřejná licence (General Public License)}

\end{seznamzkratek}

% Literatura
\nocite{*}
\def\refname{Literatura}
\bibliographystyle{mystyle}
\bibliography{literatura}


% Začátek příloh
%\def\figurename{Figure}%
%\prilohy

% Vysázení seznamu příloh
%\seznampriloh

% Vložení souboru s přílohami
%\chapter{Pseudokód - Výpočet statistik}
\begin{algorithm}
\caption{Výpočet statistik}
\label{alg:computeStat}
	\begin{algorithmic}[1] \STATE{\textbf{funkce}
vypočtiStatistiky (poleBody)} \STATE{kumulativníČas = 0}
\STATE{časSegmentPředchozí = 0} \STATE{časMimoRastr = 0}
\STATE{délkaMimoRastr = 0} \STATE{početBodůRastr = 0}
\STATE{kumulativníDávka = 0} \STATE{hodnotaRastrPředchozí = 0}
\STATE{průměrnýPříkon = 0} \STATE{délkaTrasy = 0}
\STATE{maximálníPříkon = 0}

		\FOR{i = řada od 0 do délka(poleBody)} \IF{i <
délka(poleBody) - 1} \STATE{vzdálenost =
vypočtiVzdálenost(poleBody[i], poleBody[i+1])} \ELSE \STATE{vzdálenost
= 0}
			\ENDIF
			
			\STATE{délkaTrasy = délkaTrasy + vzdálenost}

			\STATE{časSegment = vzdálenost / rychlost}
\STATE{kumulativníČas = kumulativníČas + časSegmentPředchozí}
\STATE{dávkaSegment = časSegmentPředchozí $\cdot$
hodnotaRastrPředchozí} \STATE{kumulativníDávka = kumulativníDávka +
dávkaSegment} \STATE{hodnotaRastr = získejHodnotu(poleBody[i])}

		\algstore{myalg}
		\end{algorithmic}
		\end{algorithm}

		\begin{algorithm}
		\begin{algorithmic} [1] \algrestore{myalg}

			\IF{bod1 mimo rastr \textbf{or} hodnotaRastr
<= 0} \STATE{hodnotaRastr = 0} \STATE{časMimoRastr = časMimoRastr +
časSegment} \STATE{délkaMimoRastr = délkaMimoRastr + vzdálenost} \ELSE
\STATE{průměrnýPříkon = průměrnýPříkon + hodnotaRastr}
\STATE{početBodůRastr = početBodůRastr + 1}
			\ENDIF
			
			\IF{hodnotaRastr > maximálníPříkon}
\STATE{maximálníPříkon = hodnotaRastr}
			\ENDIF \STATE{přidej [hodnotaRastr,
kumulativníČas, časSegmentPředchozí, kumulativníDávka] do
poleAtributů} \STATE{hodnotaRastrPředchozí = hodnotaRastr}
\STATE{časSegmentPředchozí = časSegment}
		\ENDFOR
		
		\STATE{průměrnýPříkon = průměrnýPříkon /
početBodůRastr} \STATE{statistiky = [délkaTrasy, kumulativníČas,
délkaMimoRastr, časMimoRastr, maximálníPříkon, průměrnýPříkon,
kumulativníDávka]} \STATE{\textbf{return} poleAtributů, statistiky}
	\end{algorithmic}
\end{algorithm}


\chapter{User guide} \label{User guide}
This plugin computes the gamma radiation dose of a given route on an
interpolated map of dose rate values. The secondary product of the
plugin are basic statistics about the given route. Together, these
output values help the emergency teams to determine which route will be
the best and safest for mobile radiation monitoring teams for radiation
monitoring during nuclear disaster emergency exercises (or~during the
nuclear disaster itself). It is important to mention, that results of~the~computation, especially
those, that~are~dealing with~the~gamma radiation dose, are~estimates.
The reason of this is, that the calculations only use few of~variables,
that are dealing with this problematics. For example it does not concern
the~situation on roads, weather (wind, rain..) etc.

\section{Installation}
Best way to install the plugin is via QGIS plugin repository.
Follow these instructions:

\begin{enumerate}
\item
  Go to \texttt{Plugins} drop down menu and select
  \texttt{Manage and Install Plugins...}:


\begin{figure}[H]
\centering
\includegraphics{pictures/user_guide/install_plugin_dropdown.png}
\caption{Plugins menu.}
\end{figure}

\item At this time, this plugin is not registered in the official QGIS
repository, therefore it is required to add its home repository CTU
GeoForAll Lab for it to be visible in~the~list of plugins. In order to
this to be done, in \texttt{Settings} tab hit \texttt{Add...} button and
type \url{http://geo.fsv.cvut.cz/geoforall/qgis-plugins.xml} to~\texttt{URL} 
slot.

\textbf{Note: }It is also required to tick \texttt{Show also experimental plugins} 
since this plugin is distributed as experimental. In future, the plugin 
will be published in~the~official QGIS repository.


\begin{figure}[H]
\centering
\includegraphics[scale = 0.7]{pictures/user_guide/settings.png}
\caption{Add home repository of the plugin.}
\end{figure}



\item 
Search for \texttt{Ground\ Radiation\ Monitoring} plugin on the
\texttt{All} or \texttt{Not\ installed} tab. After selecting the plugin,
hit \texttt{Install\ plugin} button:

\begin{figure}[H]
\centering
\includegraphics[scale=0.55]{pictures/user_guide/install_search_plugin.png}
\caption{Search and install the plugin.}
\end{figure}

\item
The Ground Radiation Monitoring Plugin is ready to use with the icon
appearance in the QGIS toolbar:

\begin{figure}[H]
\centering
\includegraphics{pictures/user_guide/install_toolbar.png}
\caption{Ground Radiation Monitoring Plugin on the QGIS toolbar.}
\end{figure}

\end{enumerate}

\section{Plugin description}

\subsection{Sampling the track}\label{sampling-the-track}

The distance between track vertices can be very large, especially on
sections, that are straight (e.g. highways). If there is any kind of
peak or any significant change of dose rate value on these long straight
parts of track, it is not recognized because of the stated reason. Hence
the plugin samples the track to smaller parts and allows user to choose
the length of them.

\subsection{GUI}\label{gui}

The plugin is divided into two tabs. The first of them is \texttt{Main}
tab:

\begin{figure}[H]
\centering
\includegraphics{pictures/user_guide/gui_main.png}
\caption{The main tab of plugin.}
\end{figure}

\begin{itemize}

\item
  In main tab, the user may select a raster layer with the interpolated
  map of dose rate values. He may select a vector layer with a route.
  Both of those layer could be selected using combo boxes, which include
  available layers from the Layer panel. There is a restriction for the
  track combo box, that only the layer with a linestring type is being
  shown.
\item
  Plugin also allows user to upload files with raster and vector layers
  via buttons \texttt{Load\ raster} and \texttt{Load\ track}. With
  hitting one of these buttons, a file dialog appears. Only GDAL/OGR
  supported files are shown.
\item
  A tool button \texttt{...} that follows \texttt{Report} line lets the
  user to select a destination, where the report file will be saved.
  Then the blank space in the same line is filled with the path to
  selected destination file with \emph{.txt} suffix. Also the blank
  space in~the~\texttt{Shapefile} is automatically filled with the same
  path with a different suffix (\emph{.shp} instead of \emph{.txt}).
\item
  The destination of the created shapefile could be changed with hitting
  a tool button \texttt{...} on \texttt{Shapefile} line.
\item
  Finally, \texttt{Save} button does all the work, it starts the
  computation and saves created files to selected destinations.
\end{itemize}

The second tab of the plugin is \texttt{Settings} tab:

\begin{figure}[H]
\centering
\includegraphics{pictures/user_guide/gui_settings.png}
\caption{The settings tab.}
\end{figure}

\begin{itemize}
\item
  This tab allows the user to change input variables.

  \begin{itemize}
  
  \item
    \texttt{Dose\ rate\ units} combo box sets the units of gamma
    radiation dose rate values of the interpolated map.
  \item
    \texttt{Speed} value determines what speed the mobile team is
    driving. This value has to be given.
  \item
    \texttt{Distance\ between\ track\ vertices} value is used for
    sampling the track. If~this value is not given, the track will not
    be sampled.
  \item
    \texttt{Create\ CSV\ file} checkbox gives the user an option,
    whether to create CSV file with the same data, that are written to
    created shapefile.
  \end{itemize}
\item
  Settings tab is filled with values by default.
\end{itemize}

\subsection{Input data}\label{input-data}

\begin{figure}[H]
\centering
\includegraphics{pictures/user_guide/input.png}
\caption{Input files.}
\end{figure}

\begin{itemize}

\item
  An interpolated map of dose rate values with a format supported by
  \href{http://www.gdal.org/formats_list.html}{GDAL library}.
\item
  A monitoring linestring route with a format supported by
  \href{http://www.gdal.org/ogr_formats.html}{OGR library}.
\end{itemize}

\subsection{Output data}\label{output-data}

\begin{itemize}

\item
  A text report file containing fields:

  \begin{itemize}
  \item
    time, when was the report created;
  \item
    route information - name, monitoring speed, total monitoring time,
    total distance;
  \item
    information about the part of the track, that has no data available
    (where the~track exceeds the raster) - time, distance;
  \item
    estimated radiation values - maximum and average dose rate, total
    dose;
  \item
    plugin settings - input raster units, distance between track
    vertices.
  \item
    a static text explaining the report
  \end{itemize}
\end{itemize}

\begin{figure}[H]
\centering
\includegraphics{pictures/user_guide/report.png}
\caption{The report file.}
\end{figure}

\begin{itemize}

\item
  A shapefile with point layer containing vertices of a sampled route with
  following attributes - dose rate, cumulated time, time interval from
  previous point, cumulated dose.
\end{itemize}

\begin{figure}[H]
\centering
\includegraphics{pictures/user_guide/shapefile.png}
\caption{The attribute table.}
\end{figure}

\begin{itemize}
\item
  An optional CSV file with same values as in created shapefile.
\end{itemize}

\begin{figure}[H]
\centering
\includegraphics[scale = 0.8]{pictures/user_guide/csv.png}
\caption{The CSV file.}
\end{figure}





% Konec dokumentu
\end{document}
