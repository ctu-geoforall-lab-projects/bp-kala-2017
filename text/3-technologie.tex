\chapter{Použité technologie}
\label{3-technologie}

V~této kapitole budou popsány technologie použité pro tvorbu
softwarového nástroje, jež je předmětem této bakalářské
práce. Implementace byla provedena v~programo\-vacím jazyku Python
s~využitím grafického frameworku PyQt a QGIS API (rozhraní pro
programování aplikací).

\section{Python}
\begin{figure}[H] \centering
      \includegraphics[width=100pt]{./pictures/python.png}
      \caption[Python logo]{Python logo (zdroj: Wikimedia Commons)}
      \label{fig:python}
\end{figure}
  
Python je objektově orientovaným skriptovacím jazykem provozovatelným
téměr na každé platformě. Oceňován programátory (především
začátečníky) je pro svou jednoduchou a velice efektivní syntax,
například také díky dynamickému typování (nepožaduje specifikaci
datového typu u~proměnných). Dále jeho objektový model podporuje
polymorfismus, přetěžování operátorů a vícenásobnou dědičnost. Hlavním
%%% ML: tomu nerozumim, jak souvisi "skriptovaci" jazyk s integraci C?
důvodem, proč je zařazován mezi skriptovací jazyky je jeho integrace
s~jazykem C. Díky knihovnám Python/C API lze z~programů v~Pythonu
volat kód psaný v~C nebo naopak, pro aplikace psané v~C je možné
integrovat interpret Pythonu.  Python bývá označován jako "spustitelný
pseudokód" - syntaxe se snaží vyhnout složitým zápisům a znakům ($\$,
<<, \&\&, ?$), inspiruje se v~matematice zápisem abstraktních
algoritmů.\cite{learningPython}

Jazyk Python byl navržen koncem 80. let nizozemským počítačovým
programátorem Guido van Rossumem. První verzi (0.9.0) uveřejnil
začátkem roku 1991. Hlavní principy Pythonu, které zakladatel jazyka
prosazuje, byly shrnuty do podoby 20 aforismů (\textit{"Zen of Python,
by Tim Peters"}), např.:

\begin{itemize}

	\item \textit{Na čitelnosti záleží. (Readability counts.)}
			
	\item \textit{Chyby by nikdy neměly projít bez
povšimnutí. Jedině pokud nejsou záměrně zamlčeny. (Errors should never
pass silently. Unless explicitly silenced.)}
		
	\item \textit{Měl by existovat jeden - a pokud možno pouze
jeden - zřejmý způsob jak to udělat. (There should be one - and
preferably only one - obvious way to do it.)}
\end{itemize}

Pro výpis anglického originálu Zenu stačí do konzole Pythonu napsat \\
\texttt{$>>>$~\textbf{import}~this}. Od roku 2008 je vyvíjena řada
Pythonu 3, která se snaží naplnit právě poslední ze zmiňovaných
aforismů, kdy důraz je kladen na odstranění duplicitních programových
konstrukcí a modulů. Python 3 není plně zpětně kompatibilní s~řadou 2,
softwary napsané a využívající Python 2 postupně přecházejí na jeho
novou verzi (i QGIS s~připravovanou verzí QGIS 3.0). Pro zajímavost,
jazyk byl pojmenován podle britské komediální skupiny Monty
Python. \cite{pythonHistory}

\section{Qt}

\begin{figure}[H] \centering
      \includegraphics[width=40pt]{./pictures/qt.png}
      \caption[Qt logo]{Qt logo (zdroj: Wikimedia Commons)}
      \label{fig:python}
\end{figure}

Qt je aplikační vývojový framework umožňující multiplatformní produkci
aplikací s~grafickým uživatelským rozhraním (\zk{GUI}). Qt dále nabízí
mnoho multiplatformních vývojových nástrojů pro usnadnění vývoje
aplikací pomocí této technologie. Vedle vývojového prostředí Qt
Creator (umožňuje kompletní vývoj aplikací) a dalších, je to
samostatný nástroj pro tvorbu \zk{GUI} Qt Designer, který byl využíván
při tvorbě vytvářeného softwarového nástroje. \cite{qt}

\section{PyQt} PyQt je propojení Pythonu jako programovacího jazyka a
Qt jako frameworku poskytujícího nástroj pro tvorbu
\zk{GUI}. Kombinace Pythonu a Qt umožňuje vyvíjet aplikace, které jsou
kompatibilní s~většinou používaných platforem a otevírá dveře
začátečníkům díky jednoduchosti, přehlednosti a síle Pythonu. PyQt je
používán pro vývoj všech druhů grafických aplikací, od programů pro
vedení účetnictví přes vědci užívané softwarové nástroje pro tvorbu
vizualizací až po počítačové hry. \cite{rapidPyQt}

\section{QGIS}

\begin{figure}[H] \centering
      \includegraphics[width=100pt]{./pictures/qgis.png}
      \caption[QGIS logo]{QGIS logo (zdroj: QGIS.org blog)}
      \label{fig:qgis}
\end{figure}
 
QGIS je multiplatformní geografický informační systém (\zk{GIS})
vyvíjený jako Open Source šířený pod Obecnou veřejnou licencí GNU
(\zk{GNU GPL} - GNU General Public License). \zk{GNU GPL} zaručuje
svobodu jeho sdílení a úprav, které vedou k~implementaci nových
funkcionalit a k~jeho zdokonalení. To z~něj činí mocný nástroj
používaný ve veřejném i soukromém sektoru. QGIS je psán
v~programovacím jazyku C++, jeho grafické uživatelské rozhraní je
postaveno na knihovně Qt. Projekt QGIS započal v~roce 2002, verze 1.0
byla uveřejněna roku 2009. QGIS disponuje nepřeberným množstvím
zásuvných modulů (pluginů) rozšiřujících funkčnost softwaru. Pluginy
jsou programovány v~jazyku C++ nebo Python. \cite{masteringQgis}
