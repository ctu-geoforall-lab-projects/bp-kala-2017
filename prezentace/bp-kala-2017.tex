\documentclass{beamer}

\usepackage[czech, english]{babel}
\usepackage[utf8]{inputenc}	
\usepackage[square,sort,comma,numbers]{natbib}
\usepackage{textpos}

\usetheme{Malmoe}
\title{Zásuvný modul QGIS \\ pro pozemní monitorování radiace}
\author{Michael Kala}
\date{29. června 2017}

\begin{document}

\begin{frame}
\titlepage
\end{frame}

\begin{frame}
\frametitle{Zadání}
\section{Úvod}
\begin{itemize}

\item přímo zadání

\end{itemize}
\end{frame}

\begin{frame}
\frametitle{Motivace}
\begin{itemize}
	\item Proč jsem si to vybral (má to praktický výstup použitelný v praxi) atd., spolupráce se SÚRO (již jeden zásuvný modul předtím, map corners coordinates, s kolegyní Kulovanou)
\end{itemize}
\end{frame}


\begin{frame}
\section{Pozemní monitorování radiace}
\frametitle{Pozemní monitorování radiace}
%tady ňáký obrázky tras monitorování, mezní hodnoty tadiace
\begin{itemize}
\item přeříkat nějak zkráceně, jak se monitoruje, říct o vytipovaných trasách
\end{itemize}
\end{frame}

\begin{frame}
\section{Technologie}
\frametitle{Technologie}
\begin{itemize}
	\item QGIS
	\item Python + PyQt
\end{itemize}
\end{frame}

\begin{frame}
\section{Zásuvný modul}
\frametitle{Zásuvný modul}
\begin{itemize}
	 \item Návrh, impementace, testování, Vstup (+ ukázky), výstup (+ ukázky, říct, že výstupy jsou přibližné hodnoty z důvodů zjednoduešení výpočtu, slouží informativně), práce s modulem, vzorkovaní, dokumentace, během implementace probíhalo testování na úřadě, byly zapracovány připomínky
\end{itemize}
\end{frame}

\begin{frame}
\frametitle{Vzorkování trasy}
\end{frame}

\begin{frame}
\section{Závěr}
\frametitle{Závěr}
\begin{itemize}
	\item Shrnutí (rekapitulace - co vzniklo, co je můj přínos, vznikl tento program, který se může používat takto a takto) Plán do budoucna - rozšíření o vlastní plánování trasy, v čem by se dal modul ještě použít, osobní závěr (poděkování za vstřícnost ústavu, za vedení atd.), co mi to dalo odborně 
\end{itemize}
\end{frame}

\begin{frame}
\frametitle{Otázky oponent}
\end{frame}

\end{document}